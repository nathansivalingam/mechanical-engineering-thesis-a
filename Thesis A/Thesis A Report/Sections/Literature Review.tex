\newpage
\section{Literature Review}
% \textit{What is the problem to be solved, and it's significance?
% \begin{itemize}
%     \item Brief background to project
%     \item Summary of literature relevant to project
%     \item Identification of "gaps" in the literature
% \end{itemize}}

% The literature review is not just about presenting descriptions of the important papers you have found, but telling a meaningful story and, where appropriate, some critical discussion of previous findings (i.e. was another study useful but flawed?). Remember that you may read hundreds of papers/books/web pages etc., but often only about 20 or 30 are really important, and these are the ones you will mention in your literature review, which this report will be a concise version of. This section needs to flow logically, and this does not always imply that the material is chronological. By the end, the reader should have a clear appreciation of what the major work in the field was, why it is relevant to the current project, and where the unknowns and questions lie (\textbf{research gaps}) – these are the issues that you are going to address with your thesis research.

% For the purposes of this report, this section will be \textbf{12-15 pages long}. Remember to reference properly any material that you obtain from literature or other sources. If you are unsure how to discuss literature properly, find a really good review paper on your topic, or if there isn’t one, a similar topic, and you will have a good example to refer to.

\subsection{Principles of Photovoltaic Modules}
\subsubsection{What are photovoltaic modules?}
Photovoltaic modules, commonly known as solar panels, are devices that convert sunlight into electrical energy. \cite{EnelGreenPower2025PhotovoltaicModule}\cite{U.S.DEPARTMENTofENERGY2025SolarBasicsb}\vspace{0.5em}

\subsubsection{What is the photoelectric effect?}
Sunlight is made up of massless particles called photons, which possess a certain amount of energy. When these photons strike the surface, they knock electrons off of it, known as photoelectrons. This is known as the photoelectric effect, shown in Figure \ref{fig:photoelectric_effect_diagram}.\vspace{0.5em}

\begin{figure}[ht]
    \centering
    \includegraphics[width=0.4\textwidth]{Figures/photoelectric_effect_diagram.png}
    \caption{Photoelectric Effect Diagram \cite{KhanAcademy2025PhotoelectricEffect}}
    \label{fig:photoelectric_effect_diagram}
\end{figure}
\FloatBarrier

\noindent The photoelectric effect will occur only if the frequency of the radiation is greater than the threshold frequency of the metal. The threshold frequency is the minimum frequency of light that causes electrons to be emitted from a material. The proportional relationship that exists between the threshold frequency and the work function is shown in Equation \ref{eq:work_function}.
\begin{equation}
    E = hf_0
    \label{eq:work_function}
\end{equation}

\noindent The work function refers to the minimum amount of energy needed to remove an electron from a metal surface. If photons with enough energy hit the surface, they can transfer their energy to the electrons allowing them to escape. If the energy of the incident photons is less than the work function, no electrons will be emitted, regardless of the intensity of the light. \cite{ScienceABC2025PhotoelectricBeginners}\vspace{0.5em}

\subsubsection{How do photovoltaic modules work?}
A photovoltaic module is made up of multiple photovoltaic cells, commonly known as solar cells. Each photovoltaic cell is made of semiconductor material, which is placed between the conductive layers. The most common semiconductor material used to make photovoltaic cells is silicon, accounting for 95 percent of photovoltaic modules sold worldwide. \cite{U.S.DEPARTMENTofENERGY2025SolarBasics} These photovoltaic cells use the photovoltaic effect to convert solar energy into electrical energy.\vspace{0.5em}

\noindent As shown in Figure \ref{fig:photovoltaic_cell_diagram}, a silicon photovoltaic cell is composed of two different layers of silicon: an n-type silicon layer, which has additional electrons, and a p-type silicon layer, which has extra spaces for the electrons, called holes.\par

\noindent ADD IN THE STUFF ABOUT DOPING (CAN BE FOUND IN BARRY'S SECOND PARAGRAPH IN 2.1.)

\begin{figure}[ht]
    \centering
    \includegraphics[width=1\textwidth]{Figures/photovoltaic_cell_diagram.jpg}
    \caption{Photovoltaic Cell Structure Diagram \cite{Gupta2020SolarVehicle}}
    \label{fig:photovoltaic_cell_diagram}
\end{figure}
\FloatBarrier

\noindent When a proton strikes the silicon photovoltaic cell with the required energy, an electron is knocked out of its bond. Because of the electric field at the p-n junction, the negatively charged electron moves toward the n-side, while the resulting positively charged hole is attracted to the p-side. The free electrons are collected by thin metal fingers positioned at the top of the photovoltaic cell, before travelling through an external circuit. After electrical work is performed, the electrons are returned to the conductive aluminium sheet positioned at the back of the photovoltaic cell. Since electrons follow a continuous cycle and are the only moving components, there is no wear and tear, allowing photovoltaic cells to last for decades. \cite{TED-Ed2025HowKomp} However, they still have limitations.\vspace{0.5em}

\subsection{Limitations of Photovoltaic Modules}
\subsubsection{What are the limitations of Photovoltaic Modules?}
A major limitation of photovoltaic modules is their electrical efficiency, particularly at high temperatures. A review led by Swapnil Dubey from Nanyang Technological University stated that a standard photovoltaic module typically converts 6-20 percent of incoming solar radiation into electrical energy. The remaining 80-94 percent is primarily converted into heat, which raises the module's temperature and further reduces its efficiency. \cite{Dubey2013TemperatureReview}\vspace{0.5em}

\noindent A 2010 study led by H.G. Teo from the National University of Singapore further refined the electrical efficiency range of photovoltaic modules to between 8 and 14 percent.\par 

\noindent \textbf{What did Teo et al. aim to do in this paper?}\par
\noindent - Teo et al. \cite{Teo2012AnModules} focused on comparing the electrical efficiency of the photovoltaic module with and without cooling.\par\vspace{0.5em}

\noindent \textbf{What did he find out? Did he achieve his goals/aims?}\par
\noindent Teo et al. observed that electrical efficiency decreases linearly with photovoltaic module temperature, as shown in Figure \ref{fig:electrical_efficiency_vs_temperature_pv_module}. They also found that without active cooling, the module temperature was significantly higher than when active cooling was applied under the same meteorological conditions. Thus, active cooling improved the module’s electrical efficiency from 8–9 percent to 12–14 percent. \cite{Teo2012AnModules}

\begin{figure}[ht]
    \centering
    \includegraphics[width=0.75\textwidth]{Figures/electrical_efficiency_vs_temperature_pv_module.png}
    \caption{Electrical efficiency as a function of PV temperature \cite{Teo2012AnModules}}
    \label{fig:electrical_efficiency_vs_temperature_pv_module}
\end{figure}
\FloatBarrier

\noindent\textbf{Why did Teo et al.'s results show that an increase in the temperature of the module causes a decline in the module's electrical efficiency?}\par
\noindent - A 2022 study, led by Ali.O.M. Maka of Heriot-Watt University confirmed Teo et al.'s results regarding the inverse linear relationship between the temperature of the photovoltaic module and its electrical efficiency.\par
\noindent - Maka et al. focused on the following electrical performance parameters:\par 
\begin{itemize}
    \item Short current density, $J_\text{sc,i}$
    \item Total current density, $J,i$
    \item Open circuit voltage, $V_\text{OC,i}$
    \item Fill Factor, FF
    \item Maximum Power, $P_\text{max}$
    \item Cell efficiency, $\eta$
\end{itemize}
\noindent - Maka et al. found that the band gap energy of the semiconductor materials reduces as the temperature of the photovoltaic module increases.\vspace{0.5em}

\noindent \textbf{Understanding Band Gap}\par
\noindent - The Bohr model describes an atom as having a central nucleus comprised of protons and neutrons that is surrounded by orbiting electrons.\par
\noindent - The orbits surrounding the nucleus are grouped into energy levels known as shells.\par
\noindent - The outermost shell is called the valence shell.\par
\noindent - The valence shell of an atom represents a band of energy levels, which is why it is also known as a valence band.\par
\noindent - Valence electrons are confined to the valence band.\par
\noindent - When valence electrons get enough energy from an external source, they can escape from the valence band and jump to the conduction band.\par
\noindent - The amount of energy that an electron must possess to jump from the valence band to the conduction band is called the band gap.\par\vspace{0.5em}

\noindent \textbf{How does the temperature of the photovoltaic module affect the band gap of the semiconductor material?}\par
\noindent - As the temperature of the photovoltaic module increases, the valence electrons in the semiconductor material are excited.\par
\noindent - Therefore, the energy required for an electron to transition from the valence band to the conduction band decreases, resulting in a reduction of the band gap.\par\vspace{0.5em}

\noindent \textbf{How does a reduced band gap affect the electrical efficiency of the photovoltaic module?}\par
\noindent A 1994 study led by P. Baruch from the University of Paris derived a formula for determining the minimum value of the diode saturation current, as presented in Equation \ref{eq:minimum_diode_saturation_current}.

\begin{equation}
    J_0 = \frac{q}{k} \frac{15\sigma}{\pi^4} T^3 \int_{u}^{\infty} \frac{x^2}{e^x - 1}\,dx
    \label{eq:minimum_diode_saturation_current}
\end{equation}

\noindent where $u = \frac{E_G}{kT}$. \cite{Baruch1995OnConversion}\par\vspace{0.5em}

\noindent Just over a decade later, Michael Y. Levy and Christiana Honsberg from the University of Delaware proposed a method to evaluate the integral in Baruch's formula. \cite{Levy2006RapidApplications}\par\vspace{0.5em}
\noindent In 2017, Honsberg worked with Stuart Bowden to graph the relationship of the diode saturation current and the band gap based on this method, shown in Figure \ref{fig:diode_saturation_current_bandgap_graph}.\par

\begin{figure}[ht]
    \centering
    \includegraphics[width=0.75\textwidth]{Figures/diode_saturation_current_bandgap_graph.png}
    \caption{Diode saturation current as a function of band gap. The values are determined from detailed balance and place a limit on the open circuit voltage of a solar cell. \cite{Honsberg2025Open-CircuitVoltage}}
    \label{fig:diode_saturation_current_bandgap_graph}
\end{figure}
\FloatBarrier

\noindent Honsberg and Bowden were then able to use the diode saturation current, $J_0$, denoted in Equation \ref{eq:open_circuit_voltage_function} as $I_L$, to calculate the open circuit voltage.\par

\begin{equation}
    V_{OC} = \frac{n k T}{q} \times \ln\left(\frac{I_L}{I_0} + 1\right)
    \label{eq:open_circuit_voltage_function}
\end{equation}

\noindent Thus, they were able to conclude that the open-circuit voltage is linearly proportional to the band gap, shown in Figure \ref{fig:open-circuit_voltage_bandgap_graph}. \cite{Honsberg2025Open-CircuitVoltage}\par

\begin{figure}[ht]
    \centering
    \includegraphics[width=0.75\textwidth]{Figures/open-circuit_voltage_bandgap_graph.png}
    \caption{$V_\text{OC}$ as a function of band gap for a cell with AM 0 and AM 1.5. The $V_\text{OC}$ increases with bang gap as the recombination current falls. There is drop off in $V_\text{OC}$ at very high band gaps due to the very low $I_\text{SC}$. \cite{Honsberg2025Open-CircuitVoltage}}
    \label{fig:open-circuit_voltage_bandgap_graph}
\end{figure}
\FloatBarrier

\noindent Maka et al. identified open-circuit voltage as a main performance parameter of electrical efficiency. Therefore, the band gap reduction due to an increase in the temperature of the photovoltaic module results in a reduction in the module's electrical efficiency.\par\vspace{0.5em}

% \noindent - Temperature coefficient of solar panels\par
% \noindent - Temperature coefficient: For every degree increase in temperature, solar panels lose a percentage of their power output\par
% \noindent - Higher T reduces Voltage generated by the solar panel\par
% \noindent - There is a slight increase in current but the overall power still decreases\par

% \noindent - As the temperature of the module increases, atoms in the semiconductor lattice experience more vibrations.\par
% \noindent - The expansion due to the atomic vibrations weakens the bonding forces between atoms.\par
% \noindent - As a result, the energy required for electrons to jump from the valence band to the construction band is reduced, indicating a reduction in band-gap energy.\par
% \noindent - A reduction in band-gap energy leads to an increase in dark-saturation current, $I_{\text{O}}$.\par

\subsubsection{How do we reduce the impact of these limitations?}
By finding ways to cool the module temperature.\par

\pagebreak
\subsection{Heat Transfer in Photovoltaic Modules}
% \subsection{Heat Transfer in Photovoltaic Modules}
% \begin{itemize}
%     \item What is heat transfer?
%     \item What are the types of heat transfer?
%     \item What is the fundamental concept of heat transfer?
%     \item What are some convection principles used to enhance heat transfer in a photovoltaic module?
% \end{itemize}

\subsubsection{Conduction} % How does conduction transfer heat in pv modules
\noindent What is conduction?\par
\noindent How can conduction be used to cool down photovoltaic modules?\par
\noindent Why are they not the optimal source of heat transfer for cooling down photovoltaic module?\par

\subsubsection{Radiation} % How does radiation transfer heat in pv modules
\noindent What is radiation?\par
\noindent How can radiation be used to cool down photovoltaic modules?\par
\noindent Why are they not the optimal source of heat transfer for cooling down photovoltaic module?\par

\subsubsection{Convection} % How does convection transfer heat in pv modules
\paragraph{Natural Convection and the Relevant Cooling Methods (Passive Cooling)} % What role does natural convection play in the convective heat transfer in pv modules
\paragraph{Forced Convection and the Relevant Cooling Methods (Active Cooling)} % What role does natural convection play in the convective heat transfer in pv modules
\subsubsection{Vortex Generators: Vortex Induced Heat Induction} % How are vortices used to induce heat induction

\pagebreak
\subsection{Experimental Techniques}
\subsubsection{Infrared Thermography}
\subsubsection{Thermocouple Sensors}
\subsubsection{Particle Image Velocimetry}

\pagebreak
\subsection{Literature Gap}

\pagebreak


% \subsubsection{Vortex Induced Heat Transfer}
% \begin{itemize}
%     \item What is a vortex?
%     \item How does a vortex/system of vortexes induce heat transfer?
%     \item Are there any drawbacks of vortex-induced heat transfer? If so, what are they?
% \end{itemize}

% \subsubsection{Conduction}
% \begin{itemize}
%     \item What is conduction?
%     \item How can conduction be used to enhance heat transfer?
%     \item What are some examples of this principle in practice?
%     \item What were the results of this practice?
%     \item Are there any drawbacks of conduction as a method to enhance heat transfer?
% \end{itemize}

% \subsubsection{Radiation}
% \begin{itemize}
%     \item What is radiation?
%     \item How can radiation be used to enhance heat transfer?
%     \item What are some examples of this principle in practice?
%     \item What were the results of this practice?
%     \item Are there any drawbacks of radiation as a method to enhance heat transfer?
% \end{itemize}

% \subsection{Convective Photovoltaic Module Cooling Methods}
% \begin{itemize}
%     \item What is convection?
%     \item What are some cooling methods for convective photovoltaic modules?
% \end{itemize}

% \subsubsection{Cooling through Natural Convection}
% \begin{itemize}
%     \item What is Natural Convection?
%     \item How is natural convection used as a cooling method for photovoltaic modules?
%     \item Are there drawbacks to natural convection as a cooling method for photovoltaic modules?
%     \item How does forced convection compare to natural convection as a cooling method for photovoltaic modules?
% \end{itemize}

% \subsubsection{Cooling through Forced Convection}
% \begin{itemize}
%     \item What is forced convection?
%     \item How is forced convection used as a cooling method for photovoltaic modules?
%     \begin{itemize}
%         \item DC Fan Experiment
%         \item Floating Photovoltaic Module Experiment
%     \end{itemize}
%     \item Are there drawbacks to forced convection as a cooling method for photovoltaic modules?
%     \item Air vs Water Cooling Experiment
%     \item Air Cooled Modified Photovoltaic Module Experiment
%     \item Single Fin vs Multiple Fin Experiment
% \end{itemize}

% \subsubsection{Cooling through Vortex Generators}
% \begin{itemize}
%     \item What is a vortex generator?
%     \item What is the purpose of vortex generators?
%     \item Examples of Vortex Generator Experiments in the Context of Photovoltaic Module Cooling
% \end{itemize}