% Define the UNSW logo and title of the thesis here, including author and supervisor names
\title{
    \includegraphics[width=0.5\textwidth]{Figures/unsw_crest.jpg}\\ \vspace{10mm}
    \textbf{MMAN4951 - UG Thesis A\\ 
    \textit{Interim Report and Project Plan\vspace{6mm}}}
    } 
\author{Author: Nathan Sivalingam, z5359644 \vspace{2mm} \\ Supervisor: Dr Charitha de Silva\vspace{4mm} \\ UNSW Sydney, School of Mechanical \& Manufacturing Engineering}
\date{25/04/2025} % Can include the date of submission here

% Creates the title defined above
\maketitle

% The abstract can be written here
\begin{abstract}
    Write a short abstract outlining the importance of the project, your progress to date, and your planned future work. Do not exceed 100 words in the abstract. The maximum length of the entire report is 20 pages, excluding appendices. This will form a strong foundation for your final Research Thesis report. Please regard this as a great opportunity to collect your initial thoughts and data. You can use Microsoft Word or other document editors such as LaTeX, InDesign, or OpenOffice, however font sizes and margins should be similar to this guide. NOTE: this guide is indicative and can be modified in agreement with your supervisor. Please refer to the course outline (marking criteria and rubrics) for more details about expectations for each section.
\end{abstract}

% The nomenclature section is created here (variables and definitions can be edited below).
\nomenclature[1A]{\(A\)}{amplitude of …}
\nomenclature[1Cp]{\(C_p\)}{…}
\nomenclature[1D]{\(D\)}{…}
\nomenclature[1alpha]{\(\alpha\)}{…}
\nomenclature[2TLA]{\(TLA\)}{…}
\nomenclature[3PV]{\(PV\)}{Photovoltaic}

\printnomenclature

% The students email is inserted below
\footnotetext[1]{contact email}
