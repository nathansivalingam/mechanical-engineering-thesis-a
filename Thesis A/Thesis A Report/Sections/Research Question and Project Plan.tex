\section{Research Question and Project Plan}

This section will start with a clear statement on your research question, i.e. what you want to discover in relation to the already available literature and its gaps (connect to previous section). Hypothesis and aims at the basis of your research will also be presented to detail your research question, again in relation to what has been already observed in literature (e.g. a particular aspect is not considered because multiple studies have shown it is not relevant). After detailing your research question, you will describe with technical detail how you are going to conduct your research (research plan). In particular you should discuss:

% Use itemize to create dot points
\begin{itemize}
    \item your proposed solution/experimental methodology to address the research question;
    \item your thesis timeline (possibly with a Gantt chart or some kind of dated mindmap, which can go in appendix, and can be referenced to in this section) and a justification of time allocation for each task;
    \item the resources you have identified as available to your research; and
    \item the required training and upskilling you will need to obtain.
\end{itemize}
Try to corredate textual descriptions with visual aids (e.g. pictures of your experimental rig).
This section is \textbf{3-5 pages} long. 